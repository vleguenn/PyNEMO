
% ================================================================
% INTRODUCTION
% ================================================================

\chapter{Introduction}

Implementing a model configuration that is limited to an oceanic region or a basin
is being increasingly desired within the NEMO community. In doing so, the BDY 
sub-routines can be employed to communicate information from outside the regional
model domain into the interior while supporting both outflow and inflow conditions.
This set of tools was born out of a requirement to have a generic method by which 
to provide boundary data for use by these sub-routines.The original code for these
tools was written in Mathworks Matlab. It was subsequently translated into Python 
for wider distribution and to facilitate development. It is far easier to integrate 
existing scripts into Python, should the need arise.In thier current form these 
tools are by no means generic and polished, but it is hoped form a foundation from
which something more formal can be developed if the desire within the community 
exists.In the following section there is a summary of usage, along with summary 
output from two examples that are included with the code.


\subsubsection{Pre v3.4 NEMO}

Prior to v3.4 NEMO the BDY code relies on time stamp information within the BDY
files. For example if a simulation for March 2000 is run with open boundaries 
supplied with daily mean data, the BDY code requires an input file with 33 time entries:
with a corresponding time\_counter equal to midday 29 February through to midday
1 April. In v3.4 NEMO time stamp information is discarded with assumed equal time spacing
employed. In the example above the BDY input files, if seperated monthly, would consist of 
three files. One each for February, March and April containing 29, 31 and 30 time entries.
However, this becomes an issue if using say 5 day means. For example at the end of March in 
above example, the last 5 day mean in March may be centred on 28 at 12:00, thus the BDY 
takes the first entry in April to complete the boundary forcing data for March, interpolating
in time between the two points. This would imply the 5 day mean in April is centred on
2 at 12:00. All well and good. However, when the simulation is continued for April
the first time entry in the April file is now assumed to be centred on 3 at 12:00 ($\delta t/2$
into the month where $\delta t$ is the meaning period of input data. So there can be upto
4 days missmatch in this example. Hence when using source data >daily mean < monthly mean
all destination BDY files are linearly interpolated onto daily means to aviod this 
mismatch. Pre v3.4 BDY files are provided as monthly files with an additional time entry
for the previous/proceeding month if required. If concatenating pre v3.4 files later, care 
should be given to the handling of addition time entries to avoid duplications.

\subsubsection{Changes between releases}
The NEMO BDY tools are current at alpha release and thus far have had no major
changes.\\

$\bullet$ The main modifications are :\\
\begin{enumerate}
\item none as of yet
\end{enumerate}


\subsubsection{Future additions?}
Ideas for possible future development of the python tools

$\bullet$ Additions to code :\\
\begin{enumerate}
\item Multiple tide model inputs
\item Handle more than monthly boundary output files
\item Read in existing bdy indices to allow match up with existing simulations
\item B/C grid option
\item Tool to generate NEMO style mesh/mask files to allow use of other model
      data as input
\item Allow input data to be on a generic vertical grid (z-level only at the moment)
\item netcdf classic output files (currently only v4)
\item have I accounted for EW wrap?
\item GUI or generic method to provide dst\_msk
\item temporal spacing in bdy files 
\item convert matlab plot routines to interogate BDY output versus nearest neightbour 
      source file
\item optional output frequency
\item optional temporal smoothing
\item optional spatial smoothin - only 1-2-1 at the moment
\item scale barotropic velocity by src\_dep over dst\_dep to maintain volume transport
\end{enumerate}

